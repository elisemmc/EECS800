\documentclass[11pt]{amsart}
\usepackage{geometry}                % See geometry.pdf to learn the layout options. There are lots.
\geometry{letterpaper}                   % ... or a4paper or a5paper or ... 
%\geometry{landscape}                % Activate for for rotated page geometry
%\usepackage[parfill]{parskip}    % Activate to begin paragraphs with an empty line rather than an indent
\usepackage{graphicx}
\graphicspath{ {images/} }
\usepackage{subfig}
\usepackage{amssymb}
\usepackage{epstopdf}
\DeclareGraphicsRule{.tif}{png}{.png}{`convert #1 `dirname #1`/`basename #1 .tif`.png}

\title{Support Vector Machines}
\author{Elise McEllhiney}
\date{\today}                                           % Activate to display a given date or no date

\begin{document}
\maketitle

\section{Final Outputs}
\begin{align}
\text{Support Vectors:} \quad & set([0, 2, 4])\\
\text{Slack Vectors:} \quad & set([4, 6])
\end{align}
\\
\section{Explanation}

The code first finds the weight function given the input points.  This is calculated as:
$$\pmb{w} = \sum_i^n \alpha_i y_i \pmb{x}_i$$

Using the primal weight vector I can calculate the gammas as follows:
$$\gamma_j = \Big\lvert \frac{ \pmb{w}^T \pmb{x}_j + b }{ \lVert \pmb{x} \rVert} \Big\rvert$$

If the gamma value of the point is within the specified tolerance of the minimum gamma value, then the vector is considered a support vector.  The support vectors are the points that lie on the edge of our margin.  We use a tolerance since floating point numbers don't produce exact answers when computed.  The point is a slack vector if it fails to meet the following condition:
$$y_j * \pmb{w}^T \pmb{x}_j + b \geq 1$$

The slack vectors are the points that are misclassified by our data.

\end{document}